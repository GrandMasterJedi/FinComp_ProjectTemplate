%%%%%%%%%%%%%%%%%%%%%%%%%%%%%%%%%%%%%%%%%
% Homework Assignment Article
% LaTeX Template
% Version 1.3.5r (2018-02-16)
%
% This template has been downloaded from:
% /cl.uni-heidelberg.de/~zimmermann/
%
% Original author:
% Victor Zimmermann (zimmermann@cl.uni-heidelberg.de)
%
% License:
% CC BY-SA 4.0 (https://creativecommons.org/licenses/by-sa/4.0/)
%
%%%%%%%%%%%%%%%%%%%%%%%%%%%%%%%%%%%%%%%%%

%----------------------------------------------------------------------------------------

\documentclass[a4paper,11pt]{article} % Uses article class in A4 format

%----------------------------------------------------------------------------------------
%	FORMATTING
%----------------------------------------------------------------------------------------

\setlength{\parskip}{0pt}
\setlength{\parindent}{0pt}
\setlength{\voffset}{-15pt}

%----------------------------------------------------------------------------------------
%	PACKAGES AND OTHER DOCUMENT CONFIGURATIONS
%----------------------------------------------------------------------------------------

\usepackage[a4paper, margin=2.5cm]{geometry} % Sets margin to 2.5cm for A4 Paper
\usepackage[onehalfspacing]{setspace} % Sets Spacing to 1.5

\usepackage[T1]{fontenc} % Use European encoding
\usepackage[utf8]{inputenc} % Use UTF-8 encoding
\usepackage{charter} % Use the Charter font
\usepackage{microtype} % Slightly tweak font spacing for aesthetics

\usepackage[english, ngerman]{babel} % Language hyphenation and typographical rules

\usepackage{amsthm, amsmath, amssymb} % Mathematical typesetting
\usepackage{marvosym, wasysym} % More symbols
\usepackage{float} % Improved interface for floating objects
\usepackage[final, colorlinks = true, 
            linkcolor = black, 
            citecolor = black,
            urlcolor = black]{hyperref} % For hyperlinks in the PDF
\usepackage{graphicx, multicol} % Enhanced support for graphics
\usepackage{xcolor} % Driver-independent color extensions
\usepackage{rotating} % Rotation tools
\usepackage{listings, style/lstlisting} % Environment for non-formatted code, !uses style file!
\usepackage{pseudocode} % Environment for specifying algorithms in a natural way
\usepackage{style/avm} % Environment for f-structures, !uses style file!
\usepackage{booktabs} % Enhances quality of tables

\usepackage{tikz-qtree} % Easy tree drawing tool
\tikzset{every tree node/.style={align=center,anchor=north},
         level distance=2cm} % Configuration for q-trees
\usepackage{style/btree} % Configuration for b-trees and b+-trees, !uses style file!

\usepackage{titlesec} % Allows customization of titles
\renewcommand\thesection{\arabic{section}.} % Arabic numerals for the sections
\titleformat{\section}{\large}{\thesection}{1em}{}
\renewcommand\thesubsection{\alph{subsection})} % Alphabetic numerals for subsections
\titleformat{\subsection}{\large}{\thesubsection}{1em}{}
\renewcommand\thesubsubsection{\roman{subsubsection}.} % Roman numbering for subsubsections
\titleformat{\subsubsection}{\large}{\thesubsubsection}{1em}{}

\usepackage[all]{nowidow} % Removes widows

\usepackage[backend=biber,style=numeric,
            sorting=nyt, natbib=true]{biblatex} % Complete reimplementation of bibliographic facilities
\addbibresource{main.bib}
\usepackage{csquotes} % Context sensitive quotation facilities

\usepackage[yyyymmdd]{datetime} % Uses YEAR-MONTH-DAY format for dates
\renewcommand{\dateseparator}{-} % Sets dateseparator to '-'

\usepackage{fancyhdr} % Headers and footers
\pagestyle{fancy} % All pages have headers and footers
\fancyhead{}\renewcommand{\headrulewidth}{0pt} % Blank out the default header
\fancyfoot[L]{\textsc{Group Assignment}} % Custom footer text
\fancyfoot[C]{} % Custom footer text
\fancyfoot[R]{\thepage} % Custom footer text

\newcommand{\note}[1]{\marginpar{\scriptsize \textcolor{red}{#1}}} % Enables comments in red on margin

%----------------------------------------------------------------------------------------

\begin{document}

%----------------------------------------------------------------------------------------
%	TITLE SECTION
%----------------------------------------------------------------------------------------

\title{Group Project - Guidelines} % Article title
\fancyhead[C]{}
\begin{minipage}{0.295\textwidth} % Left side of title section
\raggedright
Financial Computing\\ % Your lecture or course
\footnotesize % Authors text size
%\hfill\\ % Uncomment if right minipage has more lines
Yujia HU % Your name, your matriculation number
\medskip\hrule
\end{minipage}
\begin{minipage}{0.4\textwidth} % Center of title section
\centering 
\large % Title text size
Group Project\\ % Assignment title and number
\normalsize % Subtitle text size
Instructions\\ % Assignment subtitle
\end{minipage}
\begin{minipage}{0.295\textwidth} % Right side of title section
\raggedleft
\today\\ % Date
\footnotesize % Email text size
%\hfill\\ % Uncomment if left minipage has more lines
%technet@cl.uni-heidelberg.de% Your email
\medskip\hrule
\end{minipage}

%----------------------------------------------------------------------------------------
%	ARTICLE CONTENTS
%----------------------------------------------------------------------------------------
\vspace{1cm}
\section{\textbf{Requirements}}
\begin{itemize}
	\item Each team shall consist of \textbf{min 2 and max 4 team members}. The recommended size is 3.
	\item Each team needs to submit a \textbf{formal report} in \textsl{.pdf}, with programming code in the appendix. 
	\item Each team needs to submit the \textbf{project repository} in a \textsl{.zip} file. The project repository must include the project source code. If the repository is publicly posted, then the team only needs to write the on-line link in the submission message (and the \textsl{.pdf} as an attachment).
	\item The project report needs to be \textbf{min 2 and max 10 pages} in \textsl{A4}, excluding the appendix. (A shorter report does not mean lower quality.)
	\item The group composition needs to be announced at least two weeks before the submission deadline.
	\item Deadline for submission: To be announced (around week 14)
\end{itemize}	

\vspace{0.5cm}
\section{\textbf{Rubric Points}}
\begin{itemize}
	\item {[10\%] Idea: Novelty of the proposed topic and/or suitable referencing of previous work}
	\item {[50\%] Focus and contents: }
	\begin{itemize}
		\item Identify the essence of the topic
		\item Progression from problem to analysis and conclusion
	\end{itemize}

	\item {[20\%] Style: Clarity, structure, grammar and neatness of the project report}
	\item {[20\%] Technical level of the computation and the analysis (code and algorithm)}

\end{itemize}

\vspace{0.5cm}
\section{\textbf{Topics}}
Each team needs to identify the topic of the project. Team members are encouraged to meet, discuss and propose original topics. The project topic can \textbf{be or not be} within the 3 macro-topics that are subject of this course that are (1) `'Fixed Income`', (2) `'Portfolio Theory`' and (3) `'Option Pricing`'. However the topic needs to be within the general subject of finance.
Example of topics are the following:
\begin{itemize}
	\item Risk parity portfolios. Implement portfolios with equal risk contribution.
	\item Mortgage calculator
	\item Analysis of trading strategies involving stocks and interest rates
	\item Re-implementation of any of stock market volatility models / interest rate term structure model (advanced)
	\item Re-implementation of any of option pricing methods when return process does not follow the Black and Scholes assumptions
\end{itemize}

\vspace{0.5cm}
\section{\textbf{Required and Recommended Report Items}}
\begin{itemize}
	\item {[Required] Title, authors (with student ID and contact information) and date on the first page of the report}
	\item {[Recommended] Short abstract (no more than 10 lines) }
	\item {[Required] Introduction of the topic, citing most related work you referenced}
	\item {[Required] Core description of model you implemented with formulation. A reader shall be able to re-implement your model by just reading this section}
	\item {[Recommended] Description of data (simulation or real data) and demonstration results}
	\item {[Recommended] Reflection and conclusion}
	\item {[Required] Commented programming code (in Appendix) }
\end{itemize}

Additionally to the report, it is recommended that each team write a short markdown (\textsl{.md}) document to be included in the project repository. Such document shall include any software installation or execution instructions and a summary of the package. It needs to be shorter than the report and part of it can be copied from the project report.



\bigskip

%----------------------------------------------------------------------------------------
%	REFERENCE LIST
%----------------------------------------------------------------------------------------

%\printbibliography

%----------------------------------------------------------------------------------------

\end{document}
